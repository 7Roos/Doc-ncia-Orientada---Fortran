
\documentclass[xcolor=table]{beamer}
%%%%%%%%%%%%%%%%%%%%%%%%%%%%%%%%%%%%%%%%%%%%%%%%%%%%%%%%%%%%%%%%%%%%%%%%%%%%%%%%%%%%%%%%%%%%%%%%%%%%%%%%%%%%%%%%%%%%%%%%%%%%%%%%%%%%%%%%%%%%%%%%%%%%%%%%%%%%%%%%%%%%%%%%%%%%%%%%%%%%%%%%%%%%%%%%%%%%%%%%%%%%%%%%%%%%%%%%%%%%%%%%%%%%%%%%%%%%%%%%%%%%%%%%%%%%
\usepackage{mathpazo}
\usepackage{hyperref}
\usepackage{multimedia}
\usepackage[brazilian]{babel}
\usepackage{graphicx}
%Tabela estilizada
\definecolor{Maroon}{cmyk}{0,0.87,0.68,0.32}
\setlength{\arrayrulewidth}{0.9mm}
\setlength{\tabcolsep}{9pt}
\renewcommand{\arraystretch}{1.8}

%TCIDATA{OutputFilter=LATEX.DLL}
%TCIDATA{Version=5.50.0.2953}
%TCIDATA{<META NAME="SaveForMode" CONTENT="1">}
%TCIDATA{BibliographyScheme=Manual}
%TCIDATA{Created=Tuesday, July 26, 2022 00:16:25}
%TCIDATA{LastRevised=Wednesday, July 27, 2022 20:01:17}
%TCIDATA{<META NAME="GraphicsSave" CONTENT="32">}
%TCIDATA{<META NAME="DocumentShell" CONTENT="Other Documents\SW\Slides - Beamer">}
%TCIDATA{CSTFile=beamer.cst}

\newenvironment{stepenumerate}{\begin{enumerate}[<+->]}{\end{enumerate}}
\newenvironment{stepitemize}{\begin{itemize}[<+->]}{\end{itemize} }
\newenvironment{stepenumeratewithalert}{\begin{enumerate}[<+-| alert@+>]}{\end{enumerate}}
\newenvironment{stepitemizewithalert}{\begin{itemize}[<+-| alert@+>]}{\end{itemize} }
\usetheme{Madrid}
\input{tcilatex}
\begin{document}
\title[Simpson 1/3]{Técnicas de integração}
\subtitle{Método de Simpson 1/3}
\author[Roos]{Matheus Roos}
\institute[UFSM]{Universidade Federal de Santa Maria}
\date{\today}
\maketitle

\begin{frame}
	\tableofcontents
\end{frame}

\section{M\'{e}todo de Newton-Cotes fechado}

%TCIMACRO{\TeXButton{BeginFrame}{\begin{frame}}}%
%BeginExpansion
\begin{frame}%
%EndExpansion

\frametitle{Defini\c{c}\~{a}o de M\'{e}todo de Newton-Cotes fechado}

\begin{itemize}
\item Defini\c{c}\~{a}o: aproximar $f$ entre $-h$ e $-h$ por uma fun\c{c}%
\~{a}o que possa ser integrada exatamente.

\item Supomos ent\~{a}o que conhecemos $f$ em todos os pontos de uma rede
igualmente espa\c{c}ada, assim%
\[
f_{n}=f\left( x_{n}\right) \text{; \ \ \ \ \ \ \ }x_{n}=nh\text{ \ \ \ }%
\left( n=0,\pm 1,\pm 2,...\right) . 
\]

\item Utilizando a aproxima\c{c}\~{a}o em s\'{e}rie de Taylor para $f$%
\begin{eqnarray*}
f\left( x\right)  &\approx &\sum_{n=0}^{\infty }\frac{\left( x-x_{0}\right)
^{n}}{n!}\left[ \frac{d^{n}f(x)}{dx^{n}}\right] _{x=x_{0}} \\
&=&f\left( x_{0}\right) +\left( x-x_{0}\right) \left( \frac{df\left(
x\right) }{dx}\right) _{x=x_{0}}+O\left( \left( x-x_{0}\right) ^{2}\right) .
\end{eqnarray*}

\item Avaliando na vizinhan\c{c}a de $x_{0}=0$%
\[
f\left( x\right) \approx f_{0}+xf^{\prime }+\frac{x^{2}}{2!}f^{\prime \prime
}+\frac{x^{3}}{3!}f^{\prime \prime \prime }+\frac{x^{4}}{4!}f^{\left(
iv\right) } + ...
\]
\end{itemize}

%TCIMACRO{\TeXButton{Transition: Box Out}{\transboxout}}%
%BeginExpansion
\transboxout%
%EndExpansion
%TCIMACRO{\TeXButton{EndFrame}{\end{frame}}}%
%BeginExpansion
\end{frame}%
%EndExpansion

%TCIMACRO{\TeXButton{BeginFrame}{\begin{frame}}}%
%BeginExpansion
\begin{frame}%
%EndExpansion

\frametitle{Graficamente}

\begin{itemize}
\item Podemos ent\~{a}o definir%
\[
f_{\pm 1}=f\left( x=\pm h\right) \approx f_{0}\pm hf^{\prime }+\frac{h^{2}}{%
2!}f^{\prime \prime }\pm \frac{h^{3}}{3!}f^{\prime \prime \prime }+O\left(
h^{4}\right) 
\]
\end{itemize}
\begin{figure}
	\includegraphics[scale=.16]{lista4_figurediferencafinita.png}
	\caption{Figura inspirada de Computational Physics (Steve K.) e \textit{tex} adaptado de Henri Menke.}
\end{figure}
%TCIMACRO{\TeXButton{Transition: Box Out}{\transboxout}}%
%BeginExpansion
\transboxout%
%EndExpansion
%TCIMACRO{\TeXButton{EndFrame}{\end{frame}}}%
%BeginExpansion
\end{frame}%
%EndExpansion

\section{Partindo dos m\'{e}todos j\'{a} conhecidos}

\subsection{Regra dos ret\^{a}ngulos}

%TCIMACRO{\TeXButton{BeginFrame}{\begin{frame}}}%
%BeginExpansion
\begin{frame}%
%EndExpansion

\frametitle{Regra dos ret\^{a}ngulos}

\begin{stepitemize}
\item Come\c{c}amos supondo que $f$ possa ser aproximada por uma constante
nos subintervalos de tamanho $h$, vide figura de exemplo.%
\[
f\left( x\right) \approx f\left( x_{0}\right) +\mathcal{O}\left( x\right)
=f_{0}+\mathcal{O}\left( x\right) 
\]

\item Ent\~{a}o devemos integrar, neste caso, no intervalo entre $0$ e $h$:%
\[
\int_{0}^{h}f\left( x\right) dx=\int_{0}^{x_{0}}f\left( x\right) dx ,
\]

\item substitu\'{\i}mos a aproxima\c{c}\~{a}o,%
\[
\int_{0}^{h}f\left( x\right) dx\approx \int_{0}^{h}\left[ f_{0}+\mathcal{O}\left(
x\right) \right] dx ,
\]

\item resultando em%
\begin{equation}
\int_{0}^{h}f\left( x\right) dx\approx hf_{0}+\mathcal{O}\left( h^{2}\right) .
\label{eq.retangulo}
\end{equation}
\end{stepitemize}

%TCIMACRO{\TeXButton{Transition: Box Out}{\transboxout}}%
%BeginExpansion
\transboxout%
%EndExpansion
%TCIMACRO{\TeXButton{EndFrame}{\end{frame}}}%
%BeginExpansion
\end{frame}%
%EndExpansion

%TCIMACRO{\TeXButton{BeginFrame}{\begin{frame}}}%
%BeginExpansion
\begin{frame}%
%EndExpansion

\frametitle{Desenvolvendo a f\'{o}rmula num\'{e}rica}

\begin{stepitemize}
\item Para implementarmos computacionalmente a integra\c{c}\~{a}o atrav\'{e}%
s deste m\'{e}todo, devemos antes, submeter a f\'{o}rmula anterior no
internalo $[a,b]$ a equa\c{c}\~{a}o:%
\[
\int_{a}^{b}f\left( x\right) dx=\int_{a}^{a+h}f\left( x\right)
dx+\int_{a+h}^{a+2h}f\left( x\right) dx+...+\int_{b-h}^{b}f\left( x\right)
dx.
\]

\item Ent\~{a}o utilizamos a Eq.(\ref{eq.retangulo})%
\begin{eqnarray*}
	\int_{a}^{b}f\left( x\right) dx&=hf\left( a\right) +hf\left( a+h\right)
	+hf\left( a+2h\right) \\
	&+...+hf\left( b-h\right) + \mathcal{O}\left( h^{2}\right) ,
\end{eqnarray*}
 

\item ou ent\~{a}o,%
\[
\int_{a}^{b}f\left( x\right) dx=hf(a) + h\sum_{i=1}^{n-1}f\left( x_{i}\right) + \mathcal{O}\left( h^{2}\right) . 
\]

\item Onde $x_{i}=a + i\times h$.
\end{stepitemize}

%TCIMACRO{\TeXButton{Transition: Box Out}{\transboxout}}%
%BeginExpansion
\transboxout%
%EndExpansion
%TCIMACRO{\TeXButton{EndFrame}{\end{frame}}}%
%BeginExpansion
\end{frame}%
%EndExpansion

\subsection{Regra dos trap\'{e}zios}

%TCIMACRO{\TeXButton{BeginFrame}{\begin{frame}}}%
%BeginExpansion
\begin{frame}%
%EndExpansion

\frametitle{Regra dos trap\'{e}zios}

\begin{stepitemize}
\item Agora supomos que $f$ possa ser aproximada por uma reta nos
subintervalos $[-h,h]$%
\[
f\left( x\right) \approx f_{0}+xf^{\prime }+\mathcal{O}\left( x^{2}\right) . 
\]

\item Ent\~{a}o integramos no intervalo entre $-h$ e $h$, dividindo o
intervalo de integra\c{c}\~{a}o :%
\[
\int_{-h}^{h}f\left( x\right) dx=\int_{-h}^{0}f\left( x\right)
dx+\int_{0}^{h}f\left( x\right) dx 
\]

\item Substitu\'{\i}mos a aproxima\c{c}\~{a}o,%
\begin{eqnarray*}
\int_{-h}^{h}f\left( x\right) dx &\approx &\int_{x_{-1}}^{x_{0}}\left[
f_{0}+x\frac{f_{0}-f_{-1}}{h}+\mathcal{O}\left( x^{2}\right) \right] dx \\
&&+\int_{x_{0}}^{x_{1}}\left[ f_{0}+x\frac{f_{1}-f_{0}}{h}+\mathcal{O}\left(
x^{2}\right) \right] dx.
\end{eqnarray*}
\end{stepitemize}

%TCIMACRO{\TeXButton{Transition: Box Out}{\transboxout}}%
%BeginExpansion
\transboxout%
%EndExpansion
%TCIMACRO{\TeXButton{EndFrame}{\end{frame}}}%
%BeginExpansion
\end{frame}%
%EndExpansion

%TCIMACRO{\TeXButton{BeginFrame}{\begin{frame}}}%
%BeginExpansion
\begin{frame}%
%EndExpansion

\begin{stepitemize}
\item Resultando em%
\begin{eqnarray*}
\int_{-h}^{h}f\left( x\right) dx &\approx &\left[ 0-\left( -h\right) \right]
f_{0}+\frac{f_{0}-f_{-1}}{h}\frac{0^{2}-\left( -h\right) ^{2}}{2} \\
&&+\left( h-0\right) f_{0}+\frac{f_{1}-f_{0}}{h}\left[ \frac{h^{2}-0^{0}}{2}%
\right] +\mathcal{O}\left( h^{3}\right) .
\end{eqnarray*}

\item Chegamos em%
\begin{equation}
\int_{-h}^{h}f\left( x\right) dx\approx \frac{h}{2}\left(
f_{-1}+2f_{0}+f_{1}\right) +\mathcal{O}\left( h^{3}\right) .  \label{eq.trapezio}
\end{equation}
\end{stepitemize}

%TCIMACRO{\TeXButton{Transition: Box Out}{\transboxout}}%
%BeginExpansion
\transboxout%
%EndExpansion
%TCIMACRO{\TeXButton{EndFrame}{\end{frame}}}%
%BeginExpansion
\end{frame}%
%EndExpansion

%TCIMACRO{\TeXButton{BeginFrame}{\begin{frame}}}%
%BeginExpansion
\begin{frame}%
%EndExpansion

\frametitle{Desenvolvendo a f\'{o}rmula num\'{e}rica}

\begin{stepitemize}
\item Novamente, devemos submeter nossa f\'{o}rmula \`{a}%
\[
\int_{a}^{b}f\left( x\right) dx=\int_{a}^{a+h}f\left( x\right)
dx+\int_{a+h}^{a+2h}f\left( x\right) dx+...+\int_{b-h}^{b}f\left( x\right)
dx. 
\]

\item Mas h\'{a} um detalhe sutil, de que, agora temos a exig\^{e}ncia de $n$
subintervalos pares.%
\[
\int_{a}^{b}f\left( x\right) dx=\int_{a}^{a+2h}f\left( x\right)
dx+\int_{a+2h}^{a+4h}f\left( x\right) dx+...+\int_{b-2h}^{b}f\left( x\right)
dx.
\]

\item Ent\~{a}o utilizamos a Eq.(\ref{eq.trapezio})%
\begin{eqnarray*}
\int_{a}^{b}f\left( x\right) dx &=&\frac{h}{2}\left[ f\left( a\right)
+2f\left( a+h\right) +f\left( a+2h\right) \right] \\
&&+\frac{h}{2}\left[ f\left( a+2h\right) +2f\left( a+3h\right) +f\left(
a+4h\right) \right] \\
&&+...+\frac{h}{2}\left[ f\left( b-2h\right) +2f\left( b-h\right) +f\left(
b\right) \right] +\mathcal{O}\left( h^{3}\right) .
\end{eqnarray*}
\end{stepitemize}

%TCIMACRO{\TeXButton{Transition: Box Out}{\transboxout}}%
%BeginExpansion
\transboxout%
%EndExpansion
%TCIMACRO{\TeXButton{EndFrame}{\end{frame}}}%
%BeginExpansion
\end{frame}%
%EndExpansion

%TCIMACRO{\TeXButton{BeginFrame}{\begin{frame}}}%
%BeginExpansion
\begin{frame}%
%EndExpansion

\begin{stepitemize}
\item Agrupando os termos em comum,%
\begin{eqnarray*}
\int_{a}^{b}f\left( x\right) dx &=&\frac{h}{2}\left[ f\left( a\right)
+2f\left( a+h\right) +\left( 2\right) f\left( a+2h\right) \right]  \\
&&+\frac{h}{2}\left[ 2f\left( a+3h\right) +f\left( a+4h\right) \right]  \\
&&+...+\frac{h}{2}\left[ f\left( b-2h\right) +2f\left( b-h\right) +f\left(
b\right) \right] +\mathcal{O}\left( h^{3}\right) .
\end{eqnarray*}

\item Rearranjando,%
\[
\int_{a}^{b}f\left( x\right) dx=\frac{h}{2}\left[ f\left( a \right)
+2\sum_{i=1}^{n-1}f\left( x_{i}\right) +f\left( b \right) \right] + \mathcal{O}\left( h^{3} \right) .
\]

\item Lembrando que, $x_{i}=a+i\times h$
\end{stepitemize}

%TCIMACRO{\TeXButton{Transition: Box Out}{\transboxout}}%
%BeginExpansion
\transboxout%
%EndExpansion
%TCIMACRO{\TeXButton{EndFrame}{\end{frame}}}%
%BeginExpansion
\end{frame}%
%EndExpansion

\section{O M\'{e}todo de Simpson 1/3}

\subsection{Desenvolvendo a f\'{o}rmula}

%TCIMACRO{\TeXButton{BeginFrame}{\begin{frame}}}%
%BeginExpansion
\begin{frame}%
%EndExpansion

\frametitle{Desenvolvendo a f\'{o}rmula}

\begin{stepitemize}
\item Agora supomos que $f$ possa ser aproximada por uma par\'{a}bola nos
subintervalos $[-h,h]$%
\begin{eqnarray}
f\left( x\right)  &\approx &f_{0}+xf^{\prime }+\frac{x^{2}}{2!}f^{\prime
\prime }+\mathcal{O}\left( x^{3}\right) \nonumber  \label{eq.aproximacao} \\
&\approx&f_{0}+x\frac{f_{1}-f_{-1}}{2h}+\frac{x^{2}}{2}\frac{f_{1}-2f_{0}+f_{-1}}{%
h^{2}}+\mathcal{O}\left( x^{3}\right)  .
\end{eqnarray}

\item Ent\~{a}o integramos no intervalo entre $-h$ e $h$, substituindo a
aproxima\c{c}\~{a}o Eq.(\ref{eq.aproximacao}),%
\[
\int_{-h}^{h}f\left( x\right) dx\approx \left[ xf_{0}+\frac{x^{2}}{2}\frac{%
f_{1}-f_{-1}}{2h}+\frac{x^{3}}{3!}\frac{f_{1}-2f_{0}+f_{-1}}{h^{2}}\right]
_{-h}^{h}+\mathcal{O}\left( h^{5}\right) 
\]

\item Resultando em%
\begin{equation}
\int_{-h}^{h}f\left( x\right) dx\approx h\frac{f_{1}+4f_{0}+f_{-1}}{3}%
+\mathcal{O}\left( h^{5}\right)   \label{eq.simpson1.3}
\end{equation}
\end{stepitemize}

%TCIMACRO{\TeXButton{Transition: Box Out}{\transboxout}}%
%BeginExpansion
\transboxout%
%EndExpansion
%TCIMACRO{\TeXButton{EndFrame}{\end{frame}}}%
%BeginExpansion
\end{frame}%
%EndExpansion

%TCIMACRO{\TeXButton{BeginFrame}{\begin{frame}}}%
%BeginExpansion
\begin{frame}%
%EndExpansion

\frametitle{Desenvolvendo a f\'{o}rmula num\'{e}rica}

\begin{stepitemize}
\item Submetendo nossa f\'{o}rmula \`{a}%
\[
\int_{a}^{b}f\left( x\right) dx=\int_{a}^{a+2h}f\left( x\right)
dx+\int_{a+2h}^{a+4h}f\left( x\right) dx+...+\int_{b-2h}^{b}f\left( x\right)
dx. 
\]

\item Ent\~{a}o utilizamos a Eq.(\ref{eq.simpson1.3})%
\begin{eqnarray*}
\int_{a}^{b}f\left( x\right) dx &=&\frac{h}{3}\left[ f\left( a\right)
+4f\left( a+h\right) +f\left( a+2h\right) \right]  \\
&&+\frac{h}{3}\left[ f\left( a+2h\right) +4f\left( a+3h\right) +f\left(
a+4h\right) \right]  \\
&&+\frac{h}{3}\left[ f\left( a+4h\right) +4f\left( a+5h\right) +f\left(
a+6h\right) \right]  \\
&&+...+\frac{h}{3}\left[ f\left( b-2h\right) +4f\left( b-h\right) +f\left(
b\right) \right] +\mathcal{O}\left( h^{5}\right) .
\end{eqnarray*}
\end{stepitemize}

%TCIMACRO{\TeXButton{Transition: Box Out}{\transboxout}}%
%BeginExpansion
\transboxout%
%EndExpansion
%TCIMACRO{\TeXButton{EndFrame}{\end{frame}}}%
%BeginExpansion
\end{frame}%
%EndExpansion

%TCIMACRO{\TeXButton{BeginFrame}{\begin{frame}}}%
%BeginExpansion
\begin{frame}%
%EndExpansion

\begin{stepitemize}
\item Agrupando os termos em comum,%
\begin{eqnarray*}
\int_{a}^{b}f\left( x\right) dx &=&\frac{h}{3}\left[ f\left( a\right)
+4f\left( a+h\right) +\left( 2\right) f\left( a+2h\right) \right]  \\
&&+\frac{h}{3}\left[ 4f\left( a+3h\right) +\left( 2\right) f\left(
a+4h\right) \right]  \\
&&+\frac{h}{3}\left[ 4f\left( a+5h\right) +f\left( a+6h\right) \right]  \\
&&+...+\frac{h}{3}\left[ f\left( b-2h\right) +4f\left( b-h\right) +f\left(
b\right) \right] +\mathcal{O}\left( h^{5}\right) .
\end{eqnarray*}

\item Rearranjando,%
\begin{eqnarray*}
\int_{a}^{b}f\left( x\right) dx &=&\frac{h}{3}\left[ f\left( a\right)
+4\left( \sum_{i=1}^{n-1}f\left( x_{i}\right) \right) _{\acute{\imath}mpar}%
\right]  \\
&&+\frac{h}{3}\left[ 2\left( \sum_{i=2}^{n-2}f\left( x_{i}\right) \right)
_{par}+f\left( b\right) \right] +\mathcal{O}\left( h^{5}\right),
\end{eqnarray*}

\item onde $x_{i}=a+i\times h$.
\end{stepitemize}

%TCIMACRO{\TeXButton{EndFrame}{\end{frame}}}%
%BeginExpansion
\end{frame}%
%EndExpansion
\section{Fórmulas de ordem mais elevada}
\begin{frame}
Compléments de Mathématiques, André Angot.
\begin{center}
	\begin{tabular}{ c c c c c c c }
		\rowcolor{Maroon}
		n & $H_{0}$ & $H_{1}$ & $H_{2}$ & $H_{3}$ & $H_{4}$ & $\mathcal{O}$ \\ 
		\hline \hline
		\rowcolor{Maroon!50}
		1 & 1 & 1 & & & & $\mathcal{O}\left(h^{3}\right)$ \\  
		\rowcolor{Maroon!10}
		2 & 1/3 & 4/3 & 1/3 & & & $\mathcal{O}\left(h^{5}\right)$ \\  
		\rowcolor{Maroon!50}
		3 & 1/4 & 3/4 & 3/4 & 1/4 & & $\mathcal{O}\left(h^{5}\right)$  \\  
		\rowcolor{Maroon!10}
		4 & 7/45 & 32/45 & 12/45 & 32/45 & 7/45 & $\mathcal{O}\left(h^{7}\right)$ \\ 
		\rowcolor{Maroon!50}
		5 & 19/144 & 75/144 & 50/144 & 50/144 & $...$ & $\mathcal{O}\left(h^{7}\right)$ \\  
		\rowcolor{Maroon!10}
		6 & 41/420 & 216/420 & 27/420 & 272/420 & $...$ & $\mathcal{O}\left(h^{9}\right)$ \\ 
		\rowcolor{Maroon!50}
		7 & $ \frac{751}{8640} $ & $ \frac{3577}{8640}$ & $\frac{1323}{8640}$ & $\frac{2989}{8640}$ & $...$ & $\mathcal{O}\left(h^{9}\right)$ \\  
		\rowcolor{Maroon!10}
		8 & $ \frac{989}{14175}$ & $\frac{5888}{14175}$ & $\frac{-928}{14175}$ & $\frac{10496}{14175}$ & $...$ & $\mathcal{O}\left(h^{11}\right)$
	\end{tabular}
\end{center}
\end{frame}

\section{Aplicação}
\begin{frame}
\frametitle{Funções de Teste}
\begin{itemize}
	\item Função quadrática:
	\[ \int_{1,5}^{3} 3x^{2}dx = \left. (x^{3}) \right|_{1,5}^{3} = 23,625 \ \ \ . \]
	\item Função oscilatória:
	\[ \int_{1,5}^{3} \cos(2t)dt = \left.\frac{1}{2} sen(2t) \right|_{1,5}^{3} = -0,21026 \ \ \ . \]
	\item Função exponencial:
	\[ \int_{1,5}^{3} e^{2x}dx = \left.\frac{1}{2} e^{2x} \right|_{1,5}^{3} = 191,67192 \ \ \ . \]
	\item Função logarítmica:
	\[ \int_{1,5}^{3} \ln(2x)dx = \left. x(\ln(2) + \ln(x)) \right|_{1,5}^{3} = 2,22735 \ \ \ . \]
\end{itemize}
\end{frame}

\end{document}
