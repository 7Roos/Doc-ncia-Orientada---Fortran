\documentclass[a4paper,12pt]{extarticle}
\usepackage[top=2cm]{geometry}
\usepackage{listings}
\usepackage{xcolor}
\usepackage[brazilian]{babel}

\definecolor{codegreen}{rgb}{0,0.6,0}
\definecolor{codegray}{rgb}{0.5,0.5,0.5}
\definecolor{codepurple}{rgb}{0.58,0,0.82}
\definecolor{backcolour}{rgb}{0.95,0.95,0.92}

\lstdefinestyle{mystyle}{
	backgroundcolor=\color{backcolour},   
	commentstyle=\color{codegreen},
	keywordstyle=\color{magenta},
	numberstyle=\tiny\color{codegray},
	stringstyle=\color{codepurple},
	basicstyle=\ttfamily\footnotesize,
	breakatwhitespace=false,         
	breaklines=true,                 
	captionpos=b,                    
	keepspaces=true,                 
	numbers=left,                    
	numbersep=5pt,                  
	showspaces=false,                
	showstringspaces=false,
	showtabs=false,                  
	tabsize=2
}

\lstset{style=mystyle}
\author{Matheus Roos}
\title{8 formas de escrever um programa em Fortran}
\date{\today}

\begin{document}
	\maketitle
	\begin{abstract}
		A linguagem Fortran nos oferece diversas formas de escrevermos um programa conforme a nossa necessidade. Estes procedimentos poderão se dar na forma externa ou interna, podendo ser um subprograma de função (FUNCTION) ou uma sub-rotina (SUBROUTINE) e estes podendo serem organizados através de um módulo (MODULE). Vamos ver como proceder com estas formas aplicando para o caso do cálculo de uma raiz através do método de Newton.
	\end{abstract}
\section*{Introdução}
	O script mais simples que podemos fazer para encontrar a raiz de uma função através do método de Newton-Raphsode é este abaixo

\lstinputlisting[language=Fortran]{newton01ProgramMain.f90}
 onde concentramos todo o método dentro do programa principal, este código até faz o trabalho, mas vamos explorar outras formas de escrevê-lo através dos procedimentos externos (função e sub-rotina).
 
 \section{Funções}
 Podemos passar para um procedimento externo tanto a função que será definida pelo usuário como o cálculo da derivada desta função. Assim,
 \lstinputlisting[language=Fortran]{newton02FunctionExternal.f90}
 onde a \textsc{Function Fun} atuará como uma função externa na \textsc{Function Fprime}.
\newpage
 Analogamente, poderíamos ter passado estas funções como um procedimento interno, escrevendo
 \lstinputlisting[language=Fortran]{newton03FunctionInternal.f90}
 \newpage
\section{Sub-rotina}
 Agora nós vamos introduzir uma sub-rotina, onde todo o método estará dentro dela, restando ao programa principal apenas invocá-la através do comando \verb|Call name_subroutine(input, output)|. Como podemos ver:
 \lstinputlisting[language=Fortran]{newton04Subroutine.f90}
\newpage
 Mas nós também podemos ter procedimentos internos, dentro de um procedimento externo. Segue abaixo a adaptação:
 \lstinputlisting[language=Fortran]{newton05SubroutineWithFunction.f90}
\newpage
 Da mesma forma que a gente pode passar as funções como um procedimento interno, podemos fazê-lo com a sub-rotina, como segue 
 \lstinputlisting[language=Fortran]{newton06SubroutineInternal.f90}
\newpage
 Uma grande vantagem do uso da sub-rotina é que podemos escrevê-la num arquivo separado do programa principal e então ao compilar informamos ao gFortran o nome do arquivo que contém o programa principal, bem como o da sub-rotina. O resultado será este:\\
 
 newtonSubroutineFile1.f90:
 \lstinputlisting[language=Fortran]{newton07SubroutineTwoFiles1.f90}
 \newpage
 newtonSubroutineFile2.f90:
 \lstinputlisting[language=Fortran]{newton07SubroutineTwoFiles2.f90}
 E então digitamos no terminal:
\begin{lstlisting}[language=bash]
$ gfortran newtonSubroutineFiles1.f90 newtonSubroutineFile2.f90
\end{lstlisting}
\newpage
\section{Módulo}
Há ainda a possibilidade de acomodarmos a subrotina dentro de um módulo. Então criamos um arquivo que conterá apenas o módulo e incluímos a subrotina através da declaração \verb|conteins|. Já no programa principal, para utilizar o módulo usamos o comando \verb|USE name-module|. Segue abaixo o resultado:\\

{newton08Module1.f90
 \lstinputlisting[language=Fortran]{newton08Module1.f90}
 \newpage
newtonSubroutineFile2.f90:
{newton08Module2.f90
\lstinputlisting[language=Fortran]{newton08Module2.f90}
e então compilamos no terminal:
\begin{lstlisting}[language=bash]
$ gfortran newton08Module1.f90 newton08Module2.f90
\end{lstlisting}
Será gerado além é do arquivo \verb|.out|, também um \verb|.mod|, 
\end{document}
