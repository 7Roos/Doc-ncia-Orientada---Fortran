
\documentclass[notes=show]{beamer}
%%%%%%%%%%%%%%%%%%%%%%%%%%%%%%%%%%%%%%%%%%%%%%%%%%%%%%%%%%%%%%%%%%%%%%%%%%%%%%%%%%%%%%%%%%%%%%%%%%%%%%%%%%%%%%%%%%%%%%%%%%%%%%%%%%%%%%%%%%%%%%%%%%%%%%%%%%%%%%%%%%%%%%%%%%%%%%%%%%%%%%%%%%%%%%%%%%%%%%%%%%%%%%%%%%%%%%%%%%%%%%%%%%%%%%%%%%%%%%%%%%%%%%%%%%%%
\usepackage{mathpazo}
\usepackage{hyperref}
\usepackage{multimedia}
\usepackage[brazilian]{babel}
\usepackage{graphicx}
%TCIDATA{OutputFilter=LATEX.DLL}
%TCIDATA{Version=5.50.0.2953}
%TCIDATA{<META NAME="SaveForMode" CONTENT="1">}
%TCIDATA{BibliographyScheme=Manual}
%TCIDATA{Created=Saturday, July 23, 2022 16:41:09}
%TCIDATA{LastRevised=Sunday, July 24, 2022 04:46:30}
%TCIDATA{<META NAME="GraphicsSave" CONTENT="32">}
%TCIDATA{<META NAME="DocumentShell" CONTENT="Other Documents\SW\Slides - Beamer">}
%TCIDATA{CSTFile=beamer.cst}

\newenvironment{stepenumerate}{\begin{enumerate}[<+->]}{\end{enumerate}}
\newenvironment{stepitemize}{\begin{itemize}[<+->]}{\end{itemize} }
\newenvironment{stepenumeratewithalert}{\begin{enumerate}[<+-| alert@+>]}{\end{enumerate}}
\newenvironment{stepitemizewithalert}{\begin{itemize}[<+-| alert@+>]}{\end{itemize} }
\usetheme{Madrid}
\input{tcilatex}
\begin{document}

\title[Array - Vetorizando Operações]{Array - Vetorizando as Operações}
%\subtitle{Impressive slide presentations}
\author[Roos]{Matheus Roos}
\institute[UFSM]{Universidade Federal de Santa Maria}
\date{\today}
\maketitle
\begin{frame}
\tableofcontents
\end{frame}
\section{Motiva\c{c}\~{a}o}
\subsection{Motivação}
\begin{frame}
\frametitle{Motivação}
\begin{itemize}
	\item \textbf{Alternativa} ao uso de arquivo de dados;
	\item A sistematização da álgebra linear;
	\item Necessidade de trabalhar com dados de natureza vetorial, multidimensional;
	\item Trazendo ferramentas muito \textbf{poderosas} para trabalhar com dados;
	\item O recurso array é comum as mais diversas linguagens;
	\item Praticamente todos os cálculos em Python são \textbf{vetorizados};
	\item Arrays podem ser ferramentas extremamente poderosas,nos permitindo
	aplicar \textbf{o mesmo algoritmo} repetidamente a muitos itens de dados diferentes
	com um simples loop DO;
	\item Arrays s\~{a}o obviamente uma maneira muito mais limpa e curta de
	trabalhar com opera\c{c}\~{o}es semelhantes e repetitivas.
\end{itemize}
\end{frame}

\subsection{Introdução}
%TCIMACRO{\TeXButton{BeginFrame}{\begin{frame}}}%
%BeginExpansion
\begin{frame}%
%EndExpansion

\frametitle{Introdução}

\begin{itemize}
\item Uma matriz \'{e} um grupo de vari\'{a}veis {}{}ou constantes, todas de
mesmo tipo, referenciadas por um \'{u}nico nome;

\item Um valor individual dentro do array \'{e} chamado de \textbf{array
element};

\item O elem. \'{e} identificado pelo nome do array junto com um subscrito
apontando para o local espec\'{\i}fico dentro do array;

\item O subscrito de uma matriz \'{e} do tipo INTEGER;
\end{itemize}

%TCIMACRO{\TeXButton{Transition: Box Out}{\transboxout}}%
%BeginExpansion
\transboxout%
%EndExpansion
%TCIMACRO{\TeXButton{EndFrame}{\end{frame}}}%
%BeginExpansion
\end{frame}%
%EndExpansion

\section{Fundamentos sobre Arrays}

\subsection{Declarando Arrays}

%TCIMACRO{\TeXButton{BeginFrame}{\begin{frame}}}%
%BeginExpansion
\begin{frame}%
%EndExpansion

\frametitle{Declarando Arrays}

\begin{stepitemize}
\item Primeiro devemos declarar o tipo e n\'{u}mero de elementos de um array;

\item A raz\~{a}o disso \'{e} informar ao compilar qual o tipo de dados que
ser\~{a}o armazenados e o espa\c{c}o na mem\'{o}ria necess\'{a}rio\footnote{Alocação dinâmica de memória};

\item Declaramos uma matriz \textbf{real,} contendo \textbf{n} elementos, de
nome \textbf{name\_array}:%
\[
\text{REAL, DIMENSION(n) :: name\_array} 
\]%
\textbf{n} deve ser um inteiro, ele tamb\'{e}m \'{e} chamado de \textbf{%
extens\~{a}o da matrix}.
\end{stepitemize}

%TCIMACRO{\TeXButton{Transition: Box Out}{\transboxout}}%
%BeginExpansion
\transboxout%
%EndExpansion
%TCIMACRO{\TeXButton{EndFrame}{\end{frame}}}%
%BeginExpansion
\end{frame}%
%EndExpansion

\subsection{Estrutura de um array}

%TCIMACRO{\TeXButton{BeginFrame}{\begin{frame}}}%
%BeginExpansion
\begin{frame}%
%EndExpansion

\frametitle{Estrutura de um array}

\begin{itemize}
\item Podemos definir uma matriz de constantes;

\item Delimitadores - construtores de arrays

\item Os delimitadores em Fortra 90 s\~{a}o a (/ e /)%
\[
\text{(/ 1, 2 ,3 , 4 /)} 
\]

\item No Fortran 2003 os delimitadores s\~{a}o colchetes [ ]%
\[
\text{\lbrack\ 1, 2, 3, 4 ]} 
\]

\item Apesar dos colchetes serem um recurso trazido no Fortra 2003, o
GFortran o traz ao compilar um .f90. Veja o \textbf{Exemplo 1}.
\end{itemize}

%TCIMACRO{\TeXButton{Transition: Box Out}{\transboxout}}%
%BeginExpansion
\transboxout%
%EndExpansion
%TCIMACRO{\TeXButton{EndFrame}{\end{frame}}}%
%BeginExpansion
\end{frame}%
%EndExpansion

\section{Trabalhando com arrays}

\subsection{Opera\c{c}\~{o}es básicas}

%TCIMACRO{\TeXButton{BeginFrame}{\begin{frame}}}%
%BeginExpansion
\begin{frame}%
%EndExpansion

\frametitle{Trabalhando com arrays - Operações básicas}

\begin{itemize}
\item Cada elemento de um array \'{e} uma \textbf{vari\'{a}vel} como
qualquer outra;

\item Um elemento de array pode ser utilizado em \textbf{qualquer lugar}
onde uma vari\'{a}vel comum seria usada;

\item Elementos de matriz podem ser inclu\'{\i}dos em express\~{o}es aritm%
\'{e}ticas e l\'{o}gicas,

\item Os resultados de uma express\~{a}o podem ser atribu\'{\i}dos a um
elemento de matriz.

\item Os elementos de uma lista come\c{c}am pelo \'{\i}ndice 1, n\~{a}o pelo
0 como em outras linguagens. Vide \textbf{Exemplo 2}.

\item Podemos alterar os valores dos elementos de uma lista. Veja o \textbf{%
Exemplo 3}.
\end{itemize}

%TCIMACRO{\TeXButton{Transition: Box Out}{\transboxout}}%
%BeginExpansion
\transboxout%
%EndExpansion
%TCIMACRO{\TeXButton{EndFrame}{\end{frame}}}%
%BeginExpansion
\end{frame}%
%EndExpansion

\subsection{Pausa para trabalhar com a formata\c{c}\~{a}o}

%TCIMACRO{\TeXButton{BeginFrame}{\begin{frame}}}%
%BeginExpansion
\begin{frame}%
%EndExpansion

\frametitle{Pausa para trabalhar com a formata\c{c}\~{a}o}

\begin{itemize}
\item Podemos formatar a sa\'{\i}da dos dados, uma das formas de se fazer
isso \'{e} substituindo um n\'{u}mero no caracter correspondente a formata%
\c{c}\~{a}o.

\item No caso da declara\c{c}\~{a}o PRINT%
\begin{eqnarray*}
\text{PRINT *, value \ \ } &\rightarrow &\text{ \ \ \ \ PRINT 10, value} \\
&\rightarrow &\text{ \ \ \ \ 10 FORMAT(códigos de formata\c{c}\~{a}o)}
\end{eqnarray*}

\item No caso do READ E WRITE ser\'{a} no segundo asterisco:%
\begin{eqnarray*}
&&\text{READ(*,10)} \\
&&\text{WRITE(*,10)} \\
&&\text{10 FORMAT (códigos de formata\c{c}\~{a}o)}
\end{eqnarray*}
\end{itemize}

%TCIMACRO{\TeXButton{Transition: Box Out}{\transboxout}}%
%BeginExpansion
\transboxout%
%EndExpansion
%TCIMACRO{\TeXButton{EndFrame}{\end{frame}}}%
%BeginExpansion
\end{frame}%
%EndExpansion

%TCIMACRO{\TeXButton{BeginFrame}{\begin{frame}}}%
%BeginExpansion
\begin{frame}%
%EndExpansion

\frametitle{C\'{o}digos de formata\c{c}\~{a}o}

\begin{itemize}
\item C\'{o}digos de formata\c{c}\~{a}o:

\begin{itemize}
\item A -\TEXTsymbol{>} string de texto

\item D -\TEXTsymbol{>} N%
%TCIMACRO{\U{ba} }%
%BeginExpansion
${{}^o}$
%EndExpansion
de precis\~{a}o dupla, nota\c{c}\~{a}o exponencial

\item E -\TEXTsymbol{>} N\'{u}meros reais, nota\c{c}\~{a}o exponencial

\item F -\TEXTsymbol{>} N\'{u}meros reais, formato de ponto fixo

\item I \ -\TEXTsymbol{>} N\'{u}mero inteiro

\item X -\TEXTsymbol{>} Salto horizontal (espa\c{c}o em branco)

\item / \ -\TEXTsymbol{>} Salto vertical (quebra de linha).
\end{itemize}

\item Os c\'{o}digos D, E e F possuem a forma geral D\textit{w.d}, onde 
\textit{w}=width (largura do campo) e \textit{d}=digits (n%
%TCIMACRO{\U{ba} }%
%BeginExpansion
${{}^o}$
%EndExpansion
de d\'{\i}gitos significativos).

\item Veja o \textbf{Exemplo 4 }e \textbf{5}.
\end{itemize}

%TCIMACRO{\TeXButton{Transition: Box Out}{\transboxout}}%
%BeginExpansion
\transboxout%
%EndExpansion
%TCIMACRO{\TeXButton{EndFrame}{\end{frame}}}%
%BeginExpansion
\end{frame}%
%EndExpansion

\subsection{Loop DO impl\'{\i}cito}

%TCIMACRO{\TeXButton{BeginFrame}{\begin{frame}}}%
%BeginExpansion
\begin{frame}%
%EndExpansion

\frametitle{Loop DO impl\'{\i}cito}

\begin{stepitemizewithalert}
\item O loop DO impl\'{\i}cito tamb\'{e}m \'{e} permitido em instru\c{c}\~{o}%
es de i/o;

\item Ele vai permitir que uma lista de argumentos seja escrita muitas vezes
em fun\c{c}\~{a}o de uma vari\'{a}vel de \'{\i}ndice;

\item Cada argumento na lista de argumentos \'{e} escrito uma vez para cada
valor da vari\'{a}vel de \'{\i}ndice no loop DO impl\'{\i}cito;

\item A estrutura geral para esse loop impl\'{\i}cito \'{e} a seguinte:%
\[
\text{WRITE (unit, format) (name\_list1, index=istart, iend, istep)} 
\]

\item Onde name\_list1 s\~{a}o os valores a serem escritos/lidos;

\item index devem ser inteiros;

\item Veja o \textbf{Exemplo 6};

\item Tamb\'{e}m podemos fazer loop impl\'{\i}citos aninhados (um loop
dentro doutro). Veja o \textbf{Exemplo 7} e \textbf{8}.
\end{stepitemizewithalert}

%TCIMACRO{\TeXButton{Transition: Box Out}{\transboxout}}%
%BeginExpansion
\transboxout%
%EndExpansion
%TCIMACRO{\TeXButton{EndFrame}{\end{frame}}}%
%BeginExpansion
\end{frame}%
%EndExpansion

\subsection{Diferen\c{c}a entre loop impl\'{\i}cito e expl\'{\i}cito}

%TCIMACRO{\TeXButton{BeginFrame}{\begin{frame}}}%
%BeginExpansion
\begin{frame}%
%EndExpansion

\frametitle{Diferen\c{c}a entre loop impl\'{\i}cito e expl\'{\i}cito}

\begin{stepitemize}
\item Um loop padr\~{a}o
\begin{figure}
	\includegraphics[scale=.5]{script1}
\end{figure} 

\item se comportar\'{a} como \textbf{v\'{a}rios} WRITE/READ com alguns
argumentos

\begin{figure}
	\includegraphics[scale=.5]{script2}
\end{figure} 

\item J\'{a} com o loop impl\'{\i}cito,
\begin{figure}
	\includegraphics[scale=.5]{script3}
\end{figure}

\item teremos \textbf{um} WRITE/READ para muitos argumentos
\begin{figure}
	\includegraphics[scale=.5]{script4}
\end{figure}

\item Al\'{e}m da formata\c{c}\~{a}o de sa\'{\i}da. Vejamos como isso ocorre
com o \textbf{Exemplo 9}.
\end{stepitemize}

%TCIMACRO{\TeXButton{Transition: Box Out}{\transboxout}}%
%BeginExpansion
\transboxout%
%EndExpansion
%TCIMACRO{\TeXButton{EndFrame}{\end{frame}}}%
%BeginExpansion
\end{frame}%
%EndExpansion

\subsection{Se\c{c}\~{a}o de Array}

%TCIMACRO{\TeXButton{BeginFrame}{\begin{frame}}}%
%BeginExpansion
\begin{frame}%
%EndExpansion

\frametitle{Se\c{c}\~{a}o de Array}

\begin{itemize}
\item Podemos selecionar fatias de uma lista, ou seja, se\c{c}\~{o}es;

\item Escolhendo um start e um stop, tal que, \textbf{start:stop};

\item Podemos come\c{c}ar a contar a partir de um certo valor, se for a
partir do 2%
%TCIMACRO{\U{ba} }%
%BeginExpansion
${{}^o}$
%EndExpansion
elem., ent\~{a}o \textbf{2:}

\item Se ent\~{a}o quisermos contar at\'{e} um certo valor, se for at\'{e} o
5%
%TCIMACRO{\U{ba} }%
%BeginExpansion
${{}^o}$
%EndExpansion
elem., ent\~{a}o \textbf{:5}.\textbf{\ }Vejamos o \textbf{Exemplo 10};

\item Podemos escolher o passo com que \'{e} percorrida uma matriz, da forma 
\textbf{start:stop:step}

\item Ao chamarmos um array e atribuirmos um valor a ele, toda a matriz
assumir\'{a} esse valor.Vejamos o \textbf{Exemplo 11};

\item Tamb\'{e}m podemos percorrer os valores de forma regressiva, ou seja,
com um step negativo;
\end{itemize}

%TCIMACRO{\TeXButton{Transition: Box Out}{\transboxout}}%
%BeginExpansion
\transboxout%
%EndExpansion
%TCIMACRO{\TeXButton{EndFrame}{\end{frame}}}%
%BeginExpansion
\end{frame}%
%EndExpansion

\section{Uso do Array}

\subsection{Quando Utilizar Arrays?}

%TCIMACRO{\TeXButton{BeginFrame}{\begin{frame}}}%
%BeginExpansion
\begin{frame}%
%EndExpansion

\frametitle{Quando Utilizar Arrays?}

\begin{itemize}
\item Como podemos decidir se faz ou n\~{a}o sentido usar um array em um
problema espec\'{\i}fico?

\item Em geral, se muitos ou todos os dados de entrada devem estar na mem%
\'{o}ria ao mesmo tempo para resolver um problema de forma eficiente, ent%
\~{a}o o uso de arrays para armazenar esses dados ser\'{a} apropriado para
este fim;

\item Caso contr\'{a}rio, as matrizes n\~{a}o ser\~{a}o necess\'{a}rias;

\item Um programa para calcular a m\'{e}dia e o desvio padr\~{a}o n\~{a}o
precisam de matrizes;

\item Mas para encontrar a mediana de um conjunto de dados requer que os
dados sejam ordenados de forma crescente. Como a ordena\c{c}\~{a}o requer
que todos os dados estejam na mem\'{o}ria, um programa que calcula a mediana
deve usar uma matriz para armazenar todos os dados de entrada antes do in%
\'{\i}cio dos c\'{a}lculos.
\end{itemize}

%TCIMACRO{\TeXButton{Transition: Box Out}{\transboxout}}%
%BeginExpansion
\transboxout%
%EndExpansion
%TCIMACRO{\TeXButton{EndFrame}{\end{frame}}}%
%BeginExpansion
\end{frame}%
%EndExpansion

\subsection{Quando N\~{a}o Utilizar Array}

%TCIMACRO{\TeXButton{BeginFrame}{\begin{frame}}}%
%BeginExpansion
\begin{frame}%
%EndExpansion

\frametitle{Quando N\~{a}o Utilizar Array}

\begin{itemize}
\item O que h\'{a} de errado em usar um array dentro de um programa mesmo
que n\~{a}o seja necess\'{a}rio?

\begin{stepenumerate}
\item Matrizes desnecess\'{a}rias desperdi\c{c}am mem\'{o}ria. Arrays
desnecess\'{a}rios podem consumir muita mem\'{o}ria, tornando um programa
maior do que o necess\'{a}rio. Um programa grande requer mais mem\'{o}ria
para execut\'{a}-lo, o que torna o computador em que ele roda mais caro. Em
alguns casos, o tamanho extra pode impossibilitar a execu\c{c}\~{a}o em um
computador espec\'{\i}fico de alguma forma.

\item Matrizes desnecess\'{a}rias restringem os recursos do programa. Para
entender este ponto, vamos considerar um programa que calcula a m\'{e}dia e
o desvio padr\~{a}o de um conjunto de dados. Se o programa for projetado com
uma matriz de entrada est\'{a}tica de 1.000 elementos, ele funcionar\'{a}
apenas para conjuntos de dados com at\'{e} 1.000 elementos. Se encontrarmos
um dado conjunto com mais de 1000 elementos, o programa teria que ser
recompilado e revinculado com um tamanho de matriz maior. Por outro lado, um
programa que calcula a m\'{e}dia e o desvio padr\~{a}o de um conjunto de
dados \`{a} medida que os valores s\~{a}o inseridos n\~{a}o tem limite
superior no tamanho do conjunto de dados.
\end{stepenumerate}
\end{itemize}

%TCIMACRO{\TeXButton{Transition: Box Out}{\transboxout}}%
%BeginExpansion
\transboxout%
%EndExpansion
%TCIMACRO{\TeXButton{EndFrame}{\end{frame}}}%
%BeginExpansion
\end{frame}%
%EndExpansion


\end{document}
