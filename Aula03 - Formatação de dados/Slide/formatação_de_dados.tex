
\documentclass[notes=show]{beamer}
%%%%%%%%%%%%%%%%%%%%%%%%%%%%%%%%%%%%%%%%%%%%%%%%%%%%%%%%%%%%%%%%%%%%%%%%%%%%%%%%%%%%%%%%%%%%%%%%%%%%%%%%%%%%%%%%%%%%%%%%%%%%%%%%%%%%%%%%%%%%%%%%%%%%%%%%%%%%%%%%%%%%%%%%%%%%%%%%%%%%%%%%%%%%%%%%%%%%%%%%%%%%%%%%%%%%%%%%%%%%%%%%%%%%%%%%%%%%%%%%%%%%%%%%%%%%
\usepackage{mathpazo}
\usepackage{hyperref}
\usepackage{multimedia}
\usepackage[brazilian]{babel}
\usepackage{graphicx}
\usepackage[rightcaption]{sidecap}
%TCIDATA{OutputFilter=LATEX.DLL}
%TCIDATA{Version=5.50.0.2953}
%TCIDATA{<META NAME="SaveForMode" CONTENT="1">}
%TCIDATA{BibliographyScheme=Manual}
%TCIDATA{Created=Monday, July 18, 2022 22:42:16}
%TCIDATA{LastRevised=Wednesday, July 20, 2022 18:47:48}
%TCIDATA{<META NAME="GraphicsSave" CONTENT="32">}
%TCIDATA{<META NAME="DocumentShell" CONTENT="Other Documents\SW\Slides - Beamer">}
%TCIDATA{CSTFile=beamer.cst}

\newenvironment{stepenumerate}{\begin{enumerate}[<+->]}{\end{enumerate}}
\newenvironment{stepitemize}{\begin{itemize}[<+->]}{\end{itemize} }
\newenvironment{stepenumeratewithalert}{\begin{enumerate}[<+-| alert@+>]}{\end{enumerate}}
\newenvironment{stepitemizewithalert}{\begin{itemize}[<+-| alert@+>]}{\end{itemize} }
\usetheme{Madrid}
\input{tcilatex}
\begin{document}

\title[Trabalahndo com arquivos de dados]{Trabalhando com arquivos de dados}
\subtitle{Derivada numérica de dados}
\author[Matheus Roos]{Matheus Roos}
\institute[UFSM]{Universidade Federal de Santa Maria}
\date{\today}
\maketitle
\begin{frame}
\tableofcontents
\end{frame}

\section{Motiva\c{c}\~{a}o}

%TCIMACRO{\TeXButton{BeginFrame}{\begin{frame}}}%
%BeginExpansion
\begin{frame}%
%EndExpansion

\frametitle{Motiva\c{c}\~{a}o}

\begin{itemize}
\item Situa\c{c}\~{a}o onde apenas dados s\~{a}o fornecidos;

\item \'{A}reas como a astrof\'{\i}sica recebem dados dos observat\'{o}rios;

\item Ent\~{a}o como trabalhar com os arquivos de dados de terceiros?

\item Como calcular a derivada num\'{e}rica sem um fun\c{c}\~{a}o
previamente conhecida?
\end{itemize}

\end{frame}%
%EndExpansion

\section{O Problema}

%TCIMACRO{\TeXButton{BeginFrame}{\begin{frame}}}%
%BeginExpansion
\begin{frame}%
%EndExpansion

\frametitle{O Problema}

\begin{itemize}
\item Vamos tomar como base o exemplo apresentado no livro Elementary
Mechanics Using Python, Ander Malthe-S$\varnothing$orensse:
\begin{figure}
	\includegraphics[scale=.35]{fig4-5.png}
\end{figure}

\item Onde uma bola de t\^{e}nis \'{e} solta de uma altura de
aproximadamente 1,5m a partir do repouso;

\item Vamos primeiramente fazer um plot destes dados;
\end{itemize}

\end{frame}%
%EndExpansion

\subsection{O desafio}

%TCIMACRO{\TeXButton{BeginFrame}{\begin{frame}}}%
%BeginExpansion
\begin{frame}%
%EndExpansion

\frametitle{O Desafio}

\begin{itemize}
\item Nosso interesse \'{e} a partir destes dados, tirar informa\c{c}\~{o}es
da f\'{\i}sica do fen\^{o}meno/experimento;

\item Come\c{c}amos descrevendo o movimento;

\item Definimos o deslocamento $\Delta x(t_{0})$ no intervalo de $t=t_{0}$ a 
$t=t_{0}+\Delta t=t_{1}$ como:%
\[
\Delta x(t_{0})=x(t_{0}+\Delta t)-x(t_{0})=x(t_{1})-x(t_{0})
\]
\item Vamos então calcular o deslocamento;
\end{itemize}

%TCIMACRO{\TeXButton{Transition: Box Out}{\transboxout}}%
%BeginExpansion
\transboxout%
%EndExpansion
%TCIMACRO{\TeXButton{EndFrame}{\end{frame}}}%
%BeginExpansion
\end{frame}%
%EndExpansion



\section{Arquivo de dados}

\subsection{Defini\c{c}\~{a}o}

%TCIMACRO{\TeXButton{BeginFrame}{\begin{frame}}}%
%BeginExpansion
\begin{frame}%
%EndExpansion

\frametitle{Arquivos de dados}

\begin{itemize}
\item Quando estivermos trabalhando com um grande volume de dados, devemos
optar pelo uso do recurso dos arquivos de dados em detrimento a imprimir os valores na
tela;

\item O arquivo de dados \'{e} armazenado na mem\'{o}ria secund\'{a}ria (a
mem\'{o}ria RAM \'{e} prim\'{a}ria);

\item A mem\'{o}ria secund\'{a}ria \'{e} mais lenta que a mem\'{o}ria
principal do computador, mas ainda assim, permite acesso r\'{a}pido aos
dados;

\item Nos prim\'{o}rdios da computa\c{c}\~{a}o o dispositivo de
armazenamento secund\'{a}rio eram as fitas magn\'{e}ticas;

\item Estas fitas magn\'{e}ticas armazenavam dados de forma semelhante as
fitas cassete;

\begin{figure}
	\includegraphics[scale=.12]{fita-k7}
\end{figure}
\begin{center}
	\tiny{Fonte: https://www.ehow.com.br}
\end{center}

\end{itemize}

%TCIMACRO{\TeXButton{EndFrame}{\end{frame}}}%
%BeginExpansion
\end{frame}%
%EndExpansion

%TCIMACRO{\TeXButton{BeginFrame}{\begin{frame}}}%
%BeginExpansion
\begin{frame}%
%EndExpansion

\frametitle{Unidade i/o}

\begin{itemize}
\item As fitas magnéticas eram lidas de forma sequencial, sem a possibilidade da leitura saltar para outro ponto;
	
\item Acesso sequencial vs acesso restrito;
	
\item Por raz\~{o}es hist\'{o}ria o Fortran trabalha com acesso sequencial.
\item A unidade de input/output \'{e} as vezes chamada de unidade l\'{o}gica
ou simplesmente unidade;

\item E corresponde ao primeiro asterisco de WRITE(*,*) ou READ(*,*);

\item O asterisco corresponder\'{a} a unidade padr\~{a}o de escrita/leitura;

\item O n\'{u}mero correspondente a unidade dever\'{a} ser um inteiro;

\item Existem v\'{a}rias instru\c{c}\~{o}es Fortran para manipular a
leitura/grava\c{c}\~{a}o dos dados nestes arquivos de dados;
\end{itemize}

%TCIMACRO{\TeXButton{EndFrame}{\end{frame}}}%
%BeginExpansion
\end{frame}%
%EndExpansion

\subsection{Declara\c{c}\~{o}es de i/o}

%TCIMACRO{\TeXButton{BeginFrame}{\begin{frame}}}%
%BeginExpansion
\begin{frame}%
%EndExpansion

\frametitle{Declara\c{c}\~{o}es de i/o}
\begin{center}
	\begin{tabular}{ l l }
		Declara\c{c}\~{a}o i/o & Descri\c{c}\~{a}o \\
		\hline \hline
		OPEN & Associa um arquivo com um nº espec\'{\i}fico de unidade \\
		CLOSE & Finaliza a associa\c{c}\~{a}o de um arquivo com uma unidade \\
		READ & L\^{e} dados de uma unidade \\
		WRITE & Escreve dados numa unidade \\
		REWIND & Move para o in\'{\i}cio de um arquivo de dados \\
		BACKSPACE & Volta um registro de um arquivo de dados. \\
		\hline
	\end{tabular}
\end{center}

%TCIMACRO{\TeXButton{EndFrame}{\end{frame}}}%
%BeginExpansion
\end{frame}%
%EndExpansion

\subsection{Atributos da declara\c{c}\~{a}o OPEN}

%TCIMACRO{\TeXButton{BeginFrame}{\begin{frame}}}%
%BeginExpansion
\begin{frame}%
%EndExpansion

\frametitle{A Declara\c{c}\~{a}o OPEN}

\begin{itemize}
\item A declara\c{c}\~{a}o 
\[ \text{OPEN(lista\_argumentos)}\]
\item possui diversos
atributos dispon\'{\i}veis, que ser\~{a}o acessados atrav\'{e}s da lista de
argumentos;

\item O primeiro destes atributos especifica a unidade, sendo obrigat\'{o}%
rio;
\item O segundo designa um nome do arquivo de dados, apesar de n\~{a}o ser
obrigat\'{o}rio \'{e} altamente recomendado;
\item Os demais especificar\~{a}o
como se dar\'{a} a leitura. Todos opcionais.

\end{itemize}

%TCIMACRO{\TeXButton{EndFrame}{\end{frame}}}%
%BeginExpansion
\end{frame}%
%EndExpansion

%TCIMACRO{\TeXButton{BeginFrame}{\begin{frame}}}%
%BeginExpansion
\begin{frame}%
%EndExpansion
\frametitle{Atributos da declaração OPEN}
Os atributos acessíveis através da lista de argumentos, que já conhecemos, são:
\begin{itemize}
		\item UNIT: indica o n\'{u}mero da unidade ao qual associar o arquivo de
		dados, aceitando apenas valores inteiros:%
		\[
		\text{UNIT=numero\_inteiro} 
		\]
		
		\item FILE: especifica o nome do arquivo a ser aberto, aceitando valores do
		tipo character, escrevemos da seguinte forma:%
		\[
		\text{FILE= nome\_do\_arquivo  } 
		\]
\end{itemize}

%TCIMACRO{\TeXButton{EndFrame}{\end{frame}}}%
%BeginExpansion
\end{frame}%
%EndExpansion
\frametitle{Atributo STATUS}
\begin{frame}
Temos como novidade o atributo STATUS: define o status do arquivo a ser aberto, aceitando valores do
tipo character previamente estabelecidos, tal que,%
\[
\text{STATUS=expressao\_status} 
\]%
onde expressao\_status poder\'{a} assumir os valores:
	\begin{itemize}
		\item 'OLD': especifica que o arquivo a ser aberta j\'{a} existe;
		
		\item 'NEW': especifica que o arquivo que ser\'{a} aberto não existe. Padrão;
		
		\item 'REPLACE': se o arquivo j\'{a} existir ele substituirá, caso n\~{a}o
		exista ele irá criar um novo arquivo;
		
		\item 'SCRATCH': rascunho, sem nome de arquivo, sendo de leitura e escrita;
		
		\item 'UNKNOWN':Desconhecido.
	\end{itemize}
\end{frame}
%TCIMACRO{\TeXButton{BeginFrame}{\begin{frame}}}%
%BeginExpansion
\begin{frame}%
%EndExpansion
\frametitle{Demais atributos}
\begin{itemize}
\item ACTION:Especifica a a\c{c}\~{a}o a ser tomada ao abrir o arquivo,
apenas leitura, apenas escrita, ou ent\~{a}o leitura e escrita (padr\~{a}o),
ele aceitar\'{a} valores do tipo character pr\'{e}-estabelecidos,na forma;%
\[
\text{ACTION=expressao\_action}
\]%
expressao\_action poder\'{a} assumir apenas os seguintes valores: 'READ',
'WRITE' ou ent\~{a}o 'READWRITE' (padr\~{a}o);

\item IOSTAT: Muito importante para sinalizar erros. Se a abertura do
arquivo for conclu\'{\i}da com \^{e}xito, IOSTAT retornar\'{a} um 0, caso
haja algum erro retornar\'{a} um inteiro diferente de zero.

\item Obs: Depois que o arquivo \'{e} fechado, a unidade de i/o que estava
associada a ele fica livre para ser reatribu\'{\i}do a qualquer outro
arquivo em uma nova instru\c{c}\~{a}o 0PEN.
\end{itemize}

%TCIMACRO{\TeXButton{EndFrame}{\end{frame}}}%
%BeginExpansion
\end{frame}%
%EndExpansion

%TCIMACRO{\TeXButton{BeginFrame}{\begin{frame}}}%
%BeginExpansion
\begin{frame}%
%EndExpansion

\frametitle{A cl\'{a}usula IOSTAT}

\begin{itemize}
\item Podemos utilizar a cl\'{a}usula IOSTAT tamb\'{e}m em declara\c{c}\~{o}%
es READ;

\item Se a leitura for bem sucedida, IOSTAT retornar\'{a} zero;

\item Caso haja algum errro, ele retornar\'{a} um inteiro positivo diferente de zero;

\item Caso o final do arquivo seja atingido, IOSTAT retornar\'{a} um inteiro
negativo%
%TCIMACRO{%
%\TeXButton{TeX field}{\footnote{Apesar de existirem outras formas de fazer isso}}}%
%BeginExpansion
\footnote{Apesar de existirem outras formas de fazer isso}%
%EndExpansion
;

\item No FORTRAN 2003 foi inclu\'{\i}do o IOMSG, que ao invés de retornar um número, ele devolverá uma texto informando do que se trata o erro.

\item O programa ser\'{a} abortado caso a declara\c{c}\~{a}o READ n\~{a}o
tenha o atributo IOSTAT;
\end{itemize}

%TCIMACRO{\TeXButton{EndFrame}{\end{frame}}}%
%BeginExpansion
\end{frame}%
%EndExpansion

%TCIMACRO{\TeXButton{BeginFrame}{\begin{frame}}}%
%BeginExpansion
\begin{frame}%
%EndExpansion

\frametitle{Fluxograma}
\begin{figure}
	\includegraphics[scale=.45]{diagrama}
\end{figure}
Fonte: Stephen Chapman, Fortran 95,2003 for scientists and engineers (2007).

%TCIMACRO{\TeXButton{EndFrame}{\end{frame}}}%
%BeginExpansion
\end{frame}%
%EndExpansion

\section{Retornando para a cinem\'{a}tica}
\subsection{Definindo a velocidade}

%TCIMACRO{\TeXButton{BeginFrame}{\begin{frame}}}%
%BeginExpansion
\begin{frame}%
%EndExpansion

\frametitle{Definindo a velocidade}

\begin{itemize}
	\item A quest\~{a}o agora \'{e} determinar o qu\~{a}o r\'{a}pido o objeto
	cai, mas n\~{a}o podemos utilizar apenas o deslocamento;
	
	\item O motivo disso \'{e} que os deslocamentos se tornam menores a medida
	que diminu\'{\i}mos o intervalo de tempo;
	\item Ent\~{a}o definimos a velocidade m\'{e}dia por:%
	\[
	v_{m}(t_{1})=\frac{\Delta x}{\Delta t}=\frac{x(t_{0}+\Delta t)-x(t_{0})}{%
		\Delta t}=\frac{x(t_{1})-x(t_{0})}{\Delta t}.
	\]
	
	\item E por conseguinte, definimos a velocidade instant\^{a}nea:%
	\[
	v=\lim_{\Delta t\rightarrow 0}\frac{x(t_{0}+\Delta t)-x(t_{0})}{\Delta t}=%
	\frac{dx}{dt}
	\]
	
	\item Vamos agora calcular a velocidade m\'{e}dia a partir dos dados
	fornecidos.
\end{itemize}

%TCIMACRO{\TeXButton{Transition: Box Out}{\transboxout}}%
%BeginExpansion
\transboxout%
%EndExpansion
%TCIMACRO{\TeXButton{EndFrame}{\end{frame}}}%
%BeginExpansion
\end{frame}%
%EndExpansion
\subsection{Definindo a acelera\c{c}\~{a}o}

%TCIMACRO{\TeXButton{BeginFrame}{\begin{frame}}}%
%BeginExpansion
\begin{frame}%
%EndExpansion

\frametitle{Definindo a acelera\c{c}\~{a}o}

\begin{itemize}
\item Analogamente ao que fizemos no caso da velocidade m\'{e}dia, definimos
a acelera\c{c}\~{a}o m\'{e}dia como o qu\~{a}o r\'{a}pido a velocidade muda,
escrevemos ent\~{a}o,%
\[
a_{m}(t_{1})=\frac{\Delta v}{\Delta t}=\frac{v(t_{0}+\Delta t)-v(t_{0})}{%
\Delta t}=\frac{v(t_{1})-v(t_{0})}{\Delta t},
\]

\item E por associação, definimos a acelera\c{c}\~{a}o instant\^{a}nea:%
\[
a=\lim_{\Delta t\rightarrow 0}\frac{\Delta v}{\Delta t}=\frac{dv}{dt}.
\]
\end{itemize}

%TCIMACRO{\TeXButton{EndFrame}{\end{frame}}}%
%BeginExpansion
\end{frame}%
%EndExpansion
\begin{frame}
\frametitle{O Modelo Teórico}
\begin{itemize}
	\item Dada a natureza do problema, modelamos ele por:%
	\[
	y\left( t\right) =y_{0}+v_{0}t+\frac{1}{2}gt^{2},
	\]
	
	\item cuja equa\c{c}\~{a}o para a velocidade ser\'{a}%
	\[
	v\left( t\right) =v_{0}+gt,
	\]
	
	\item com acelera\c{c}\~{a}o igual a%
	\[
	a=g=9,81m/s^{2}\text{.}
	\]
\end{itemize}
\end{frame}
%TCIMACRO{\TeXButton{BeginFrame}{\begin{frame}}}%
%BeginExpansion
\begin{frame}%
%EndExpansion

\frametitle{Esbo\c{c}ando o algoritmo}

\begin{itemize}
\item Devemos transformar o desenvolvimento te\'{o}rico feito anteriormente
em c\'{o}digo, ent\~{a}o definimos a "velocidade m\'{e}dia num\'{e}rica" como:%
\[
\bar{v}_{i}=\frac{y_{i}-y_{i-1}}{t_{i}-t_{i-1}}=\frac{y_{i}-y_{i-1}}{\Delta t%
},
\]

\item e a acelera\c{c}\~{a}o m\'{e}dia num\'{e}rica%
\[
\bar{a}=\frac{\bar{v}_{i}-\bar{v}_{i-1}}{t_{i}-t_{i-1}}=\frac{\Delta \bar{v}%
}{\Delta t}
\]
\end{itemize}

%TCIMACRO{\TeXButton{EndFrame}{\end{frame}}}%
%BeginExpansion
\end{frame}%
%EndExpansion

\subsection{Derivada num\'{e}rica de dados}

%TCIMACRO{\TeXButton{BeginFrame}{\begin{frame}}}%
%BeginExpansion
\begin{frame}%
%EndExpansion

\frametitle{Derivada Num\'{e}rica}

\begin{itemize}
\item Devemos nos recordar que n\~{a}o conhecemos $y(t)$ para todos os
valores de $t$.

\item mas apenas os valores que foram medidos de $y(t_{i})$.

\item N\~{a}o podemos encontrar uma express\~{a}o \textbf{anal\'{\i}tica
exata} para a derivada de $y(t)$.

\item Mas podemos aproximar a velocidade instant\^{a}nea da velocidade m\'{e}%
dia, de $t_{i-1}$ a $t_{i}$:%
\[
\bar{v}_{i}\left( t_{i}\right) =\frac{y\left( t_{i}+\Delta t\right) -y\left(
t_{i}\right) }{\Delta t}\simeq v\left( t_{i}\right) =\frac{dy}{dt}.
\]

\item Adotamos o mesmo procedimento para a acelera\c{c}\~{a}o:%
\[
\bar{a}_{i}\left( t_{i}\right) =\frac{v\left( t_{i}+\Delta t\right) -v\left(
t_{i}\right) }{\Delta t}\simeq a\left( t_{i}\right) =\frac{dv}{dt}.
\]

\item Ao final deveremos comparar o modelo te\'{o}rico com o experimental.
\end{itemize}

%TCIMACRO{\TeXButton{EndFrame}{\end{frame}}}%
%BeginExpansion
\end{frame}%
%EndExpansion

\end{document}
