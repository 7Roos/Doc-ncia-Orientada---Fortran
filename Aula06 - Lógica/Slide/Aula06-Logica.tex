
\documentclass[xcolor=table]{beamer}
%%%%%%%%%%%%%%%%%%%%%%%%%%%%%%%%%%%%%%%%%%%%%%%%%%%%%%%%%%%%%%%%%%%%%%%%%%%%%%%%%%%%%%%%%%%%%%%%%%%%%%%%%%%%%%%%%%%%%%%%%%%%%%%%%%%%%%%%%%%%%%%%%%%%%%%%%%%%%%%%%%%%%%%%%%%%%%%%%%%%%%%%%%%%%%%%%%%%%%%%%%%%%%%%%%%%%%%%%%%%%%%%%%%%%%%%%%%%%%%%%%%%%%%%%%%%
\usepackage{mathpazo}
\usepackage{hyperref}
\usepackage{multimedia}
\usepackage[brazilian]{babel}
\usepackage{graphicx}

%Tabela estilizada
\definecolor{Maroon}{cmyk}{0,0.87,0.68,0.32}
\setlength{\arrayrulewidth}{0.9mm}
\setlength{\tabcolsep}{9pt}
\renewcommand{\arraystretch}{1.4}
%Redimensionando a tabela, quando vazar
\usepackage{adjustbox}

%TCIDATA{OutputFilter=LATEX.DLL}
%TCIDATA{Version=5.50.0.2953}
%TCIDATA{<META NAME="SaveForMode" CONTENT="1">}
%TCIDATA{BibliographyScheme=Manual}
%TCIDATA{Created=Sunday, July 31, 2022 13:15:55}
%TCIDATA{LastRevised=Wednesday, August 03, 2022 01:40:14}
%TCIDATA{<META NAME="GraphicsSave" CONTENT="32">}
%TCIDATA{<META NAME="DocumentShell" CONTENT="Other Documents\SW\Slides - Beamer">}
%TCIDATA{CSTFile=beamer.cst}

\newenvironment{stepenumerate}{\begin{enumerate}[<+->]}{\end{enumerate}}
\newenvironment{stepitemize}{\begin{itemize}[<+->]}{\end{itemize} }
\newenvironment{stepenumeratewithalert}{\begin{enumerate}[<+-| alert@+>]}{\end{enumerate}}
\newenvironment{stepitemizewithalert}{\begin{itemize}[<+-| alert@+>]}{\end{itemize} }
\usetheme{Madrid}
\input{tcilatex}
\begin{document}

\title[Ramificações]{Lógica e ramificações}
\subtitle{Lógica de programação e ramificações em um programa}
\author[Roos]{Matheus Roos}
\institute[UFSM]{Universidade Federal de Santa Maria}
\date{\today}
\maketitle

\section{Motiva\c{c}\~{a}o}

%TCIMACRO{\TeXButton{BeginFrame}{\begin{frame}}}%
%BeginExpansion
\begin{frame}%
%EndExpansion

\frametitle{Motiva\c{c}\~{a}o}

\begin{itemize}
\item L\'{o}gica est\'{a} presente na matem\'{a}tica e em todas as
linguagens de programa\c{c}\~{a}o (sem\^{a}ntica, l\'{o}gica, sintaxe);

\item Car\'{a}ter generalista (conhecimento aplic\'{a}vel a outras
linguagens);

\item Economizador de condicionais if-else;

\item Ferramenta a mais para contruir c\'{o}digos, pois as ramifica\c{c}\~{o}%
es s\~{a}o geralmente controladas atrav\'{e}s de valores l\'{o}gicos;

\item Loops trabalham com express\~{o}es l\'{o}gicas;

\item A vari\'{a}vel do tipo LOGICAL (booleano).
\end{itemize}

%TCIMACRO{\TeXButton{Transition: Box Out}{\transboxout}}%
%BeginExpansion
\transboxout%
%EndExpansion
%TCIMACRO{\TeXButton{EndFrame}{\end{frame}}}%
%BeginExpansion
\end{frame}%
%EndExpansion

\section{Valores l\'{o}gicos}

%TCIMACRO{\TeXButton{BeginFrame}{\begin{frame}}}%
%BeginExpansion
\begin{frame}%
%EndExpansion

\frametitle{Valores l\'{o}gicos}

\begin{stepitemize}
\item O dado de tipo l\'{o}gico permite \textbf{apenas} dois valores\footnote{Lógica paraconsistente para além do V ou F, Newton Costa} (duas
constantes):

\begin{itemize}
\item .TRUE. $\rightarrow $Verdadeiro;

\item .FALSE.$\rightarrow $Falso.
\end{itemize}

\item Eles s\~{a}o do tipo LOGICAL (booleano);%
\begin{figure}
	\includegraphics{fig1}
\end{figure}

\item Raramente vamos utilizar constantes l\'{o}gicas, mas sim, express\~{o}%
es l\'{o}gicas para controlar a execu\c{c}\~{a}o;
\end{stepitemize}

%TCIMACRO{\TeXButton{Transition: Box Out}{\transboxout}}%
%BeginExpansion
\transboxout%
%EndExpansion
%TCIMACRO{\TeXButton{EndFrame}{\end{frame}}}%
%BeginExpansion
\end{frame}%
%EndExpansion

\subsection{C\'{a}lculo l\'{o}gico}

%TCIMACRO{\TeXButton{BeginFrame}{\begin{frame}}}%
%BeginExpansion
\begin{frame}%
%EndExpansion

\frametitle{C\'{a}lculo l\'{o}gico}

\begin{stepitemize}
\item Assim como no caso aritm\'{e}tico, atribu\'{\i}mos uma declara\c{c}%
\~{a}o aos c\'{a}lculos l\'{o}gicos que ser\~{a}o executados, tal que,%
\[
\text{logical\_var\_name = logical\_expr} 
\]

\item Um operador l\'{o}gico \'{e} um operador em dados num\'{e}ricos, de
caracteres ou l\'{o}gicos que geram um resultado l\'{o}gico;

\item Existem operadores relacionais e operadores combinacionais.
\end{stepitemize}

%TCIMACRO{\TeXButton{Transition: Box Out}{\transboxout}}%
%BeginExpansion
\transboxout%
%EndExpansion
%TCIMACRO{\TeXButton{EndFrame}{\end{frame}}}%
%BeginExpansion
\end{frame}%
%EndExpansion

\subsection{Operadores relacionais}

%TCIMACRO{\TeXButton{BeginFrame}{\begin{frame}}}%
%BeginExpansion
\begin{frame}%
%EndExpansion

\frametitle{Operadores relacionais}

\begin{stepitemize}
\item S\~{a}o operadores com dois operandos num\'{e}ricos ou de caracteres
que v\~{a}o produzir um resultado l\'{o}gico;

\item O resultado depende da \textbf{rela\c{c}\~{a}o} entre estes dois
valores que est\~{a}o sendo comparados;

\item A forma geral de uma rela\c{c}\~{a}o operador relacional \'{e} a
seguinte:%
\[
\text{A op B.} 
\]

\item A e B podem ser uma:

\begin{itemize}
\item express\~{a}o aritm\'{e}tica;

\item vari\'{a}vel;

\item constante;

\item ou ent\~{a}o cadeia de caracteres(string).
\end{itemize}
\end{stepitemize}

%TCIMACRO{\TeXButton{Transition: Box Out}{\transboxout}}%
%BeginExpansion
\transboxout%
%EndExpansion
%TCIMACRO{\TeXButton{EndFrame}{\end{frame}}}%
%BeginExpansion
\end{frame}%
%EndExpansion

%TCIMACRO{\TeXButton{BeginFrame}{\begin{frame}}}%
%BeginExpansion
\begin{frame}%
%EndExpansion

\frametitle{Forma dos operadores relacionais}

\begin{stepitemize}
\item Existem duas formas de se escrever um operador relacional;

\item A forma da esquerda foi introduzida no Fortran 90, enquanto que a
outra vem de vers\~{o}es anteriores. Vejamos o Exemplo 1;

\begin{center}
	\begin{tabular}{ccl}
		\rowcolor{Maroon}
		Novo & Velho & Significado \\ 
		\hline
		\rowcolor{Maroon!50}
		== & .EQ. & Igual a? \\ 
		\rowcolor{Maroon!20}
		/= & .NE. & Diferente de \\ 
		\rowcolor{Maroon!50}
		\TEXTsymbol{>} & .GT. & Maior que \\ 
		\rowcolor{Maroon!20}
		\TEXTsymbol{>}= & .GE. & Maior ou igual a \\ 
		\rowcolor{Maroon!50}
		\TEXTsymbol{<} & .LT. & Menor que \\ 
		\rowcolor{Maroon!20}
		\TEXTsymbol{<}= & .LE. & Menor ou igual a%
	\end{tabular}
\end{center}

\item Express\~{o}es matem\'{a}ticas s\~{a}o desenvolvidas antes dos
operadores relacionais. Veja o 2;

\item Ao comparar um valor real com um inteiro, o valor inteiro ser\'{a}
convertido para real. Veja o Exemplo 3.
\end{stepitemize}

%TCIMACRO{\TeXButton{Transition: Box Out}{\transboxout}}%
%BeginExpansion
\transboxout%
%EndExpansion
%TCIMACRO{\TeXButton{EndFrame}{\end{frame}}}%
%BeginExpansion
\end{frame}%
%EndExpansion

\subsection{Operadores combinacionais}

%TCIMACRO{\TeXButton{BeginFrame}{\begin{frame}}}%
%BeginExpansion
\begin{frame}%
%EndExpansion

\frametitle{Operadores combinacionais}

\begin{stepitemize}
\item S\~{a}o operadores com um ou dois operandos l\'{o}gicos que produzem
um resultado l\'{o}gico;

\item O resultado depende da \textbf{combina\c{c}\~{a}o} entre estes dois
operandos l\'{o}gicos;

\item A forma geral de uma rela\c{c}\~{a}o operador relacional \'{e} a
seguinte:%
\[
\text{A .op. B} 
\]

\item A e B podem ser:

\begin{itemize}
\item express\~{o}es l\'{o}gicas;

\item vari\'{a}veis;

\item ou constantes.
\end{itemize}
\end{stepitemize}

%TCIMACRO{\TeXButton{Transition: Box Out}{\transboxout}}%
%BeginExpansion
\transboxout%
%EndExpansion
%TCIMACRO{\TeXButton{EndFrame}{\end{frame}}}%
%BeginExpansion
\end{frame}%
%EndExpansion

%TCIMACRO{\TeXButton{BeginFrame}{\begin{frame}}}%
%BeginExpansion
\begin{frame}%
%EndExpansion

\frametitle{Forma dos operadores combinacionais}

\begin{stepitemize}
\item Os operadores e seu respectivo significado se encontram na tabela abaixo:

\begin{table}
	\begin{center}
		\begin{adjustbox}{max width=.9\textwidth}
			\begin{tabular}{lll}
				\rowcolor{Maroon}
				Operador & Fun\c{c}\~{a}o & Retorno TRUE se \\ 
				\hline
				\rowcolor{Maroon!50}
				A .AND. B & E l\'{o}gico & A e B forem verdadeiros \\
				\rowcolor{Maroon!10} 
				A .OR. B & OU L\'{o}gico & A ou B forem verdadeiros \\ 
				\rowcolor{Maroon!50}
				A .EQV. B & Equival\^{e}ncia l\'{o}gica & A \'{e} o \textbf{mesmo} que B \\ 
				\rowcolor{Maroon!10} 
				A .NEQV. B & N\~{a}o equival\^{e}ncia l\'{o}gica & A \'{e} Verdadeiro e B 
				\'{e} Falso \\ 
				\rowcolor{Maroon!50}
				.NOT. B & Nega\c{c}\~{a}o l\'{o}gica & B \'{e} Falso%
			\end{tabular}
		\end{adjustbox}
	\end{center}
\end{table}
\item Vejamos o Exemplo 4;
\end{stepitemize}

%TCIMACRO{\TeXButton{Transition: Box Out}{\transboxout}}%
%BeginExpansion
\transboxout%
%EndExpansion
%TCIMACRO{\TeXButton{EndFrame}{\end{frame}}}%
%BeginExpansion
\end{frame}%
%EndExpansion

\section{Tabela verdade}

%TCIMACRO{\TeXButton{BeginFrame}{\begin{frame}}}%
%BeginExpansion
\begin{frame}%
%EndExpansion

\frametitle{Tabela verdade}

\begin{stepitemize}
\item Os operadores l\'{o}gicos abrem v\'{a}rias possibilidades de se
construir a l\'{o}gica de um programa;

\item Uso do artif\'{\i}cio da tabela verdade-senten\c{c}as matem\'{a}ticas;

\begin{table}
	\begin{center}
		\begin{tabular}{cccccc}
			\rowcolor{Maroon}
			A & B & .AND. & .OR. & .EQV. & .NEQV. \\ 
			\hline
			\rowcolor{Maroon!50}
			V & V & V & V & V & F \\ 
			\rowcolor{Maroon!20}
			V & F & F & V & F & V \\ 
			\rowcolor{Maroon!50}
			F & V & F & V & F & V \\ 
			\rowcolor{Maroon!20}
			F & F & F & F & V & F%
		\end{tabular}
	\end{center}
\end{table}
\end{stepitemize}

%TCIMACRO{\TeXButton{Transition: Box Out}{\transboxout}}%
%BeginExpansion
\transboxout%
%EndExpansion
%TCIMACRO{\TeXButton{EndFrame}{\end{frame}}}%
%BeginExpansion
\end{frame}%
%EndExpansion

%TCIMACRO{\TeXButton{BeginFrame}{\begin{frame}}}%
%BeginExpansion
\begin{frame}%
%EndExpansion

\frametitle{Hierarquia geral dos operadores l\'{o}gicos}

\begin{stepitemize}
\item A ordem com que os operadores em uma express\~{a}o s\~{a}o avaliados 
\'{e} a seguinte:

\begin{stepenumerate}
\item Todos os operadores aritm\'{e}ticos;

\item Todos os operadores relacionais, da esq. para a dir.;

\item Todos os operadores de nega\c{c}\~{a}o, .NOT.;

\item Todos os operadores .AND., da esq. para a dir.;

\item Todos os operadores .OR., da esq. para a dir.;

\item Todos os operadores .EQV. e .NEQV., da esq. para a dir.;
\end{stepenumerate}

\item Podemos tamb\'{e}m utilizar os par\^{e}nteses para alterar a ordem
natural.
\end{stepitemize}

%TCIMACRO{\TeXButton{Transition: Box Out}{\transboxout}}%
%BeginExpansion
\transboxout%
%EndExpansion
%TCIMACRO{\TeXButton{EndFrame}{\end{frame}}}%
%BeginExpansion
\end{frame}%
%EndExpansion

\section{Controle de ramifica\c{c}\~{o}es}

%TCIMACRO{\TeXButton{BeginFrame}{\begin{frame}}}%
%BeginExpansion
\begin{frame}%
%EndExpansion

\frametitle{Controle de ramifica\c{c}\~{o}es}

\begin{stepitemize}
\item As ramifica\c{c}\~{o}es (branches) ser\~{a}o instru\c{c}\~{o}es que
nos permitiram selecionar e executar se\c{c}\~{o}es espec\'{\i}ficas de c%
\'{o}digo (chamados de blocos) enquanto pula para outras se\c{c}\~{o}es de c%
\'{o}digo;

\item H\'{a} essencialmente dois tipos dessas instru\c{c}\~{o}es, o
construtor de bloco IF e o construtor SELECT CASE;

\item Podemos tamb\'{e}m tornar a constru\c{c}\~{a}o destes blocos mais sem%
\^{a}ntica;

\item Al\'{e}m do controle intr\'{\i}seco dos construtores IF e SELECT CASE,
podemos adicionar as intru\c{c}\~{o}es EXIT e STOP para
direcionar/interromper o fluxo.
\end{stepitemize}

%TCIMACRO{\TeXButton{Transition: Box Out}{\transboxout}}%
%BeginExpansion
\transboxout%
%EndExpansion
%TCIMACRO{\TeXButton{EndFrame}{\end{frame}}}%
%BeginExpansion
\end{frame}%
%EndExpansion

\subsection{Bloco IF - instru\c{c}\~{a}o EXIT e STOP}
%TCIMACRO{\TeXButton{BeginFrame}{\begin{frame}}}%
%BeginExpansion
\begin{frame}%
%EndExpansion

\frametitle{Bloco IF - instru\c{c}\~{a}o EXIT e STOP}

\begin{stepitemize}
\item A forma mais comum de da instru\c{c}\~{a}o IF \'{e} atrav\'{e}s da
constru\c{c}\~{a}o de um bloco, que cont\'{e}m express\~{o}es que ser\~{a}o
executadas se uma dada condi\c{c}\~{a}o for verdadeira.

\begin{itemize}
\item A declara\c{c}\~{a}o END, ou ent\~{a}o END IF, \'{e} obrigat\'{o}ria
neste caso;

\item Bem como a declara\c{c}\~{a}o THEN logo ap\'{o}s a condi\c{c}\~{a}o l%
\'{o}gica.
\end{itemize}

\item Mas esta n\~{a}o \'{e} a \'{u}nica forma, podemos construir a instru%
\c{c}\~{a}o IF em apenas uma linha tamb\'{e}m, o que \'{e} totalmente
equivalente a n\'{\i}vel de compilador;

\item Podemos tamb\'{e}m adicionar crit\'{e}rios de parada. Existem dois
tipos, s\~{a}o eles:

\begin{itemize}
\item EXIT: O programa n\~{a}o executar\'{a} o que estiver dentro do bloco
IF, saltando para a proxima linha abaixo do bloco

\item STOP: Para (interrompe) todo o programa. Deve ser usado com cautela. N%
\~{a}o \'{e} recomendado utilizar em sub-rotinas. Pode vir acompanhado de
uma mensagem (recomendado principalmente quando houverem diversos STOP).
\end{itemize}

\item Veja o Exemplo 6.
\end{stepitemize}

%TCIMACRO{\TeXButton{Transition: Box Out}{\transboxout}}%
%BeginExpansion
\transboxout%
%EndExpansion
%TCIMACRO{\TeXButton{EndFrame}{\end{frame}}}%
%BeginExpansion
\end{frame}%
%EndExpansion

\subsection{Cl\'{a}usula ELSE e ELSE IF}

%TCIMACRO{\TeXButton{BeginFrame}{\begin{frame}}}%
%BeginExpansion
\begin{frame}%
%EndExpansion

\frametitle{Cl\'{a}usula ELSE e ELSE IF}

\begin{stepitemize}
\item Por vezes, gostar\'{\i}amos tanto de executar um bloco de c\'{o}digo
quando uma condi\c{c}\~{a}o IF \'{e} verdadeira, como tamb\'{e}m outro bloco
de c\'{o}digo quando for falsa;

\item A cl\'{a}usula ELSE resolve este problema, podemos interpret\'{a}-la
como significando: sen\~{a}o;

\item Mas tamb\'{e}m em outras ocasi\~{o}es, gostar\'{\i}amos de executar
mais do que dois blocos distintos de c\'{o}digo;

\item Ou seja, quando gostar\'{\i}amos de continuar testando depois do SEN%
\~{A}O (ELSE);

\item Para isso, utilizaremos a cl\'{a}usula ELSE IF;

\item O programa percorrer\'{a} \textbf{todos os blocos} de cima para baixo
at\'{e} atingir a condi\c{c}\~{a}o verdadeira;

\item Veja o Exemplo 7.
\end{stepitemize}

%TCIMACRO{\TeXButton{Transition: Box Out}{\transboxout}}%
%BeginExpansion
\transboxout%
%EndExpansion
%TCIMACRO{\TeXButton{EndFrame}{\end{frame}}}%
%BeginExpansion
\end{frame}%
%EndExpansion

\subsection{Tornando os blocos IF mais sem\^{a}nticos}

%TCIMACRO{\TeXButton{BeginFrame}{\begin{frame}}}%
%BeginExpansion
\begin{frame}%
%EndExpansion

\frametitle{Tornando os blocos IF mais sem\^{a}nticos}

\begin{stepitemize}
\item Em alguns casos teremos muitos blocos IF-ELSE encadeados (aninhados),
o que poder\'{a} tornar dif\'{\i}cil visualizar o que est\'{a} acontecendo
ali;

\item Para resolver isso, podemos nomear os blocos IF-ELSE;

\begin{figure}
	\includegraphics[scale=.7]{fig3}
\end{figure}
\end{stepitemize}

%TCIMACRO{\TeXButton{Transition: Box Out}{\transboxout}}%
%BeginExpansion
\transboxout%
%EndExpansion
%TCIMACRO{\TeXButton{EndFrame}{\end{frame}}}%
%BeginExpansion
\end{frame}%
%EndExpansion

\begin{frame}
\begin{stepitemize}
	\item Em [nome] poderemos utilizar caracteres alfanum\'{e}ricos, desde que o
	primeiro \textbf{seja uma letra};
	
	\item \lbrack nome] deve ser \textbf{\'{u}nico} dentro de cada unidade de
	programa;
	
	\item O nome do bloco deve estar contido na declara\c{c}\~{a}o END IF;
	
	\item Nas declara\c{c}\~{a}o ELSE e ELSE IF o nome do bloco \'{e} opcional;
	
	\item Nomear os blocos tornar\'{a} a inten\c{c}\~{a}o de um programador
	explicitamente claras;
	
	\item Os nomes tamb\'{e}m poder\~{a}o auxiliar o compilador a sinalizar
	erros, pois ele sempre ir\'{a} associar com o IF mais recente.
	
	\item Veja o Exemplo 8.
\end{stepitemize}
\end{frame}
\section{Construtor SELECT CASE}

%TCIMACRO{\TeXButton{BeginFrame}{\begin{frame}}}%
%BeginExpansion
\begin{frame}%
%EndExpansion

\frametitle{Construtor SELECT CASE}

\begin{stepitemize}
\item Possui um mecanismo muito semelhante ao construtor IF, mas com uma diferen\c{c}a sutil;

\item Suporta os seguintes tipos de dados de entrada:

\begin{itemize}
\item um \'{u}nico n\'{u}mero inteiro;

\item caractere;

\item uma express\~{a}o l\'{o}gica;

\item ou um intervalo de valores.
\end{itemize}

\item Veja o Exemplo 9.
\begin{figure}
	\includegraphics[scale=.65]{fig4}
\end{figure}
\end{stepitemize}

%TCIMACRO{\TeXButton{Transition: Box Out}{\transboxout}}%
%BeginExpansion
\transboxout%
%EndExpansion
%TCIMACRO{\TeXButton{EndFrame}{\end{frame}}}%
%BeginExpansion
\end{frame}%
%EndExpansion

\section{Debugging}

%TCIMACRO{\TeXButton{BeginFrame}{\begin{frame}}}%
%BeginExpansion
\begin{frame}%
%EndExpansion

\frametitle{Debugging}

\begin{stepitemize}
\item Quanto mais complexo vai ficando o c\'{o}digo, maiores ser\~{a}o as
chances de ter erros n\~{a}o esperados:

\begin{itemize}
\item Seja erros de sintaxe, ao unir os pequenos blocos do c\'{o}digo;

\item Seja erros de tempo de execu\c{c}\~{a}o;

\item Seja tamb\'{e}m, a escrita correto, mas que n\~{a}o d\'{a} o resultado
esperado.
\end{itemize}

\item Portanto, \'{e} importante usufruir de ferramentas que verifiquem o c%
\'{o}digo, peda\c{c}o por peda\c{c}o;

\begin{itemize}
\item Uma delas \'{e} utilizar o depurador;

\item Ou ent\~{a}o, incluindo declara\c{c}\~{o}es PRINT ou WRITE em pontos
espec\'{\i}ficos do c\'{o}digo.
\end{itemize}

\item O problema do depurador, \'{e} que as instru\c{c}\~{o}es e seu uso,
depender\'{a} do compilador em espec\'{\i}fico;

\item A princ\'{\i}pio, o LINUX j\'{a} vem com um depurador, o GDB;

\item Que \'{e} mais utilizado em linguagens como C/C++, mas que pode ser
utilizado para o Fortran;

\item No editor VS CODE, h\'{a} uma exten\c{c}\~{a}o, Fortran Breakpoint
Support, do Ekibun.
\end{stepitemize}
\end{frame}

\subsection{Dicas de checagem}
\begin{frame}
\frametitle{Algumas dicas de checagem}
\begin{stepitemize}
		\item Se o problema estiver numa instru\c{c}\~{a}o IF, verique se utilizou a
		condi\c{c}\~{a}o (a l\'{o}gica) de forma correta. Express\~{o}es l\'{o}gicas
		s\~{a}o complexas, pois s\~{a}o dif\'{\i}ceis de entender, e muito f\'{a}%
		ceis para bagun\c{c}ar o c\'{o}digo;
		
		\item Utilize par\^{e}nteses extras em condi\c{c}\~{o}es relacionais para
		torn\'{a}-las mais f\'{a}ceis de se entender;
		
		\item Se suas express\~{o}es l\'{o}gicas ests\~{a}o muito grandes, considere
		divid\'{\i}-las em mais senten\c{c}as mais simples e curtas;
		
		\item Ao trabalhar com valores reais, substitua testes de igualdade, pelos
		de quase igualdade , por causa dos erros de arredondamento;
		
		\item Por exemplo, ao inv\'{e}s de teste se $x=2$, considere testar%
		\[
		\left\vert x-2\right\vert <0,0001
		\]
		
		\item Qualquer valor entre $1,9999$ e $2,0001$ ser\'{a} contemplado.
\end{stepitemize}


%TCIMACRO{\TeXButton{Transition: Box Out}{\transboxout}}%
%BeginExpansion
\transboxout%
%EndExpansion
%TCIMACRO{\TeXButton{EndFrame}{\end{frame}}}%
%BeginExpansion
\end{frame}%
%EndExpansion

\end{document}
